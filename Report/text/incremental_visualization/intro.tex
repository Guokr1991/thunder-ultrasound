%(Why do incremental? (motivation))
%(Why visualize in real-time?)
%Challenges
%Chapter outline

In our work described in the previous chapter, all input data was ready and available before the reconstruction took place. This meant that all data could be examined and taken into account while reconstructing. Another advantage was that the whole operation can be performed in "bulk" with small overhead. An alternative is incremental reconstruction where the volume is generated while the b-scans and tracking data are acquired. In this situation, one can only take previously acquired data into account, and a smaller reconstruction procedure is performed for each chunk of data as it is generated by the ultrasound and tracking system.

In this chapter, we present a novel method to incrementally reconstruct tracked ultrasound data in real-time on the GPU. Incremental reconstruction is one of the main contributions of this thesis, and while the methods in the previous chapter were only nearest-neighbor approaches, we can also obtain high-quality reconstruction using interpolation techniques from \cite{coupe2005} and \cite{trobaugh1994}. The chapter starts with how to handle different rates of b-scan and tracking data when acquired incrementally, we describe a simple pixel-nearest-neighbor scheme performed incrementally on the GPU, which is then followed by a description of the high-quality incremental reconstruction. Lastly, we describe how the volume can simultaneously be visualized while it is reconstructed.