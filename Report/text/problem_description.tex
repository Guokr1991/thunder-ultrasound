3D ultrasound reconstruction can be used to generate volume data from tracked real-time 2D ultrasound frames. Compared to other imaging modalities like MRI and CT, ultrasound is a flexible low-cost solution for generating 3D image maps of the internal organs of the human body using existing 2D ultrasound scanners. This makes ultrasound the modality of choice for intraoperative use and enables image guided surgery (e.g.\ neuro- or laparoscopic) where surgical instruments are safely navigated inside the human body.

Current CPU-based methods for 3D ultrasound reconstruction are time consuming (typically from 1 minute to 1 hour depending on the quality). The overall goal of this thesis is to use GPU-based techniques to achieve real-time (or close to real-time) reconstruction and visualization of ultrasound.

\begin{description}
	\item[Step 1)] GPU-based real-time 3D ultrasound reconstruction of freehand 2D scans using a tracked ultrasound probe (different algorithms, command-line based).
	
	\item[Step 2)] Simultaneous reconstruction and visualization of the ultrasound volume as it gets built. It would be desirable for the surgeon to see a visualization of the volume while the data is acquired and the volume is generated (real-time volume rendering and slicing).
\end{description}

Technical issues such as parallelization and memory management techniques as well as recent platforms such as OpenCL, Nvidia Fermi and recent AMD graphics cards will also be evaluated.